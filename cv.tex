%% start of file `template.tex'.
%% Copyright 2006-2013 Xavier Danaux (xdanaux@gmail.com).
%
% This work may be distributed and/or modified under the
% conditions of the LaTeX Project Public License version 1.3c,
% available at http://www.latex-project.org/lppl/.


\documentclass[11pt,a4paper,sans]{moderncv}        % possible options include font size ('10pt', '11pt' and '12pt'), paper size ('a4paper', 'letterpaper', 'a5paper', 'legalpaper', 'executivepaper' and 'landscape') and font family ('sans' and 'roman')

% moderncv themes
\moderncvstyle{classic}                             % style options are 'casual' (default), 'classic', 'oldstyle' and 'banking'
\moderncvcolor{purple}                               % color options 'blue' (default), 'orange', 'green', 'red', 'purple', 'grey' and 'black'
%\renewcommand{\familydefault}{\sfdefault}         % to set the default font; use '\sfdefault' for the default sans serif font, '\rmdefault' for the default roman one, or any tex font name
%\nopagenumbers{}                                  % uncomment to suppress automatic page numbering for CVs longer than one page

% character encoding
\usepackage[utf8]{inputenc}                       % if you are not using xelatex ou lualatex, replace by the encoding you are using
%\usepackage{CJKutf8}                              % if you need to use CJK to typeset your resume in Chinese, Japanese or Korean

% adjust the page margins
\usepackage[scale=0.77]{geometry}
%\setlength{\hintscolumnwidth}{3cm}                % if you want to change the width of the column with the dates
%\setlength{\makecvtitlenamewidth}{10cm}           % for the 'classic' style, if you want to force the width allocated to your name and avoid line breaks. be careful though, the length is normally calculated to avoid any overlap with your personal info; use this at your own typographical risks...

% personal data
\name{Jens Henrik}{Vogelius}
\title{Age 40, Danish}                               % optional, remove / comment the line if not wanted
\address{Ydingvej 40}{8752 Østbirk}{Denmark}% optional, remove / comment the line if not wanted; the "postcode city" and and "country" arguments can be omitted or provided empty
\phone[mobile]{+45~51787606}                   % optional, , remove / comment the line if not wanted
\email{jenshenrik.skuldbol@gmail.com}                               %  remove / comment the line if not wanted

\begin{document}
\makecvtitle

\section{Knowledge Areas}
\cvitem{Programming}{C#/.NET, Java, Python, AngularJS, React, SQL}
\cvitem{Development}{Git, Docker, CI/CD, Azure, Agile, Testing}
%\cvitem{Other}{Latex, Quality and Infomation Security Management Systems, GAMP 5, Audit Readiness}

\section{Work Experience}
\cventry{2022--}{Senior software developer}{Mjølner Informatics}{Silkeborg/Århus}{}{
	Change of employer due to Norlys' acquisition of Mjølner Informatics. Continuing same project from my time at Norlys.
	%\newline \textit{Main learnings: People Management, Project Management, Politics, Quality Systems, Information Security, Audits.}
}
\cventry{2021--2022}{Software Developer}{Norlys}{Silkeborg}{}{
	Developing a custom order management system designed to plan and manage the daily work of 200+ electricians doing various tasks, including maintenance, cable mapping and establishing new parts of the electrical network. A React web app scaled for desktop and tablet running on top of .NET microservices with integrations to ERP, GIS and route planning systems.
	\newline \textit{Main learnings: Scala, AWS, Microservices, Project Management.}
}
\cventry{2020--2021}{Software developer}{Isobar}{Århus}{}{
	Continuing work from my time at e2y.
	%\newline \textit{Main learnings: AngularJS, Project Management.}
}
\cventry{2019--2020}{Software developer}{e2y}{Århus}{}{
	Developing a SAP Cloud Commerce (formerly Hybris) solution for a reseller of aircraft spareparts.
	\newline \textit{Main learnings: Java/Spring, SAP Cloud Commerce.}
}
\cventry{2016--2019}{Software developer}{Vertica}{Århus}{}{
	Developing ecommerce solutions for various customers, including Sanistål, Vestas, and Coop
	\newline \textit{Main learnings: Episerver and Umbraco development, ecommerce domain knowledge.}
}
\cventry{2014--2016}{Senior software developer}{IT Minds}{Copenhagen}{}{
	Fulltime employee following graduation. Mostly same Episerver project.
	%\newline \textit{Main learnings: Customer relations, backend design, Django.}
}
\cventry{2012--2014}{Part time software developer}{IT Minds}{Copenhagen}{}{
	Multiple projects of varying size in a wide range of languages, including Java, C#, ASP 2.0 and PHP. A larger Episerver project involving a fully customized webshop for selling TV subscriptions. Includes 6 months of fulltime internship required for BA degree.
 	\newline \textit{Main learnings: Customer relations, general software development.}
}

%\section{Extracurricular Activities}
%\cventry{2018--}{App Developer}{FreezerButler}{}{}{
%	Primary frontend developer for FreezerButler developing an app to help prevent food waste by allowing you to track food you have in your fridge and suggest new recipes using your leftovers.
%	\newline \textit{Main learnings: Swift, App development processes, React Native}
%}
%\cventry{2017--2018}{Teacher}{Coding Pirates}{}{}{
%	Volunteering to teach kids aged 8 to 15 how to code. This was achieved primarily in Scratch but Javascript and Python also played a part.
%}
%\cventry{2015--2016}{Teacher}{GOTO Academy Amsterdam}{}{\textit{AngularJS}}{
%	While working for Trifork I taught several courses on AngularJS.
%	Each course took four days -- covering everything from HTML and simple bindings to interlinked components.
%}

%\cventry{2010--2014}{Teacher's Assistant}{Copenhagen University}{}{\textit{Functional Programming} and \textit{Object Oriented Engineering}}{
%	Object Oriented Engineering had little to do with actual programming, focusing instead on all the softer aspects of delivering a successful software project such as requirement specification, planning, and %documentation.
%}

\section{Education}
\cventry{2007--2009}{2 years of Ba. studies}{Copenhagen University}{Copenhagen - Denmark}{\textit{Computer Science}}{}
\cventry{2010--2014}{Ba.Eng in IT}{DTU}{Lyngby - Denmark}{\textit{Computer Science}}{}

\section{Languages}
\cvitemwithcomment{Danish}{Native}{}
\cvitemwithcomment{English}{Fluent}{}
\cvitemwithcomment{German}{Pretty rusty}{}

\section{About me}
\cvitem{}{
	I am a problem solver at heart and and seeing suboptimal solutions to problems which have better solutions immediately available easily distract me -- or at worst annoys me.
	However, especially when considering programming problems I find it is important to not strife for perfection but rather consider continuous usefulness improvements.
	This balance is the crux of project management as I see it, and I enjoy being on either side of that discussion.
	\newline I love learning new tidbits of knowledge in many different fields.
	Whether picking up guitar chords or acquiring programming knowledge, the simple act of improving overall knowledge is fulfilling to me.
	This makes me less of a specialist and more of a jack-of-all-trades, but I consider that a strength in most situations when interacting with different people from different fields.
	\newline I enjoy people and spend quite a lot of time socialising -- preferably over board games with a gin and tonic in hand.
}
\clearpage
\end{document}
